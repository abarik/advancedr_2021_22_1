% Options for packages loaded elsewhere
\PassOptionsToPackage{unicode}{hyperref}
\PassOptionsToPackage{hyphens}{url}
%
\documentclass[
]{book}
\usepackage{amsmath,amssymb}
\usepackage{lmodern}
\usepackage{ifxetex,ifluatex}
\ifnum 0\ifxetex 1\fi\ifluatex 1\fi=0 % if pdftex
  \usepackage[T1]{fontenc}
  \usepackage[utf8]{inputenc}
  \usepackage{textcomp} % provide euro and other symbols
\else % if luatex or xetex
  \usepackage{unicode-math}
  \defaultfontfeatures{Scale=MatchLowercase}
  \defaultfontfeatures[\rmfamily]{Ligatures=TeX,Scale=1}
\fi
% Use upquote if available, for straight quotes in verbatim environments
\IfFileExists{upquote.sty}{\usepackage{upquote}}{}
\IfFileExists{microtype.sty}{% use microtype if available
  \usepackage[]{microtype}
  \UseMicrotypeSet[protrusion]{basicmath} % disable protrusion for tt fonts
}{}
\makeatletter
\@ifundefined{KOMAClassName}{% if non-KOMA class
  \IfFileExists{parskip.sty}{%
    \usepackage{parskip}
  }{% else
    \setlength{\parindent}{0pt}
    \setlength{\parskip}{6pt plus 2pt minus 1pt}}
}{% if KOMA class
  \KOMAoptions{parskip=half}}
\makeatother
\usepackage{xcolor}
\IfFileExists{xurl.sty}{\usepackage{xurl}}{} % add URL line breaks if available
\IfFileExists{bookmark.sty}{\usepackage{bookmark}}{\usepackage{hyperref}}
\hypersetup{
  pdftitle={Advanced R},
  pdfauthor={Kálmán Abari},
  hidelinks,
  pdfcreator={LaTeX via pandoc}}
\urlstyle{same} % disable monospaced font for URLs
\usepackage{longtable,booktabs,array}
\usepackage{calc} % for calculating minipage widths
% Correct order of tables after \paragraph or \subparagraph
\usepackage{etoolbox}
\makeatletter
\patchcmd\longtable{\par}{\if@noskipsec\mbox{}\fi\par}{}{}
\makeatother
% Allow footnotes in longtable head/foot
\IfFileExists{footnotehyper.sty}{\usepackage{footnotehyper}}{\usepackage{footnote}}
\makesavenoteenv{longtable}
\usepackage{graphicx}
\makeatletter
\def\maxwidth{\ifdim\Gin@nat@width>\linewidth\linewidth\else\Gin@nat@width\fi}
\def\maxheight{\ifdim\Gin@nat@height>\textheight\textheight\else\Gin@nat@height\fi}
\makeatother
% Scale images if necessary, so that they will not overflow the page
% margins by default, and it is still possible to overwrite the defaults
% using explicit options in \includegraphics[width, height, ...]{}
\setkeys{Gin}{width=\maxwidth,height=\maxheight,keepaspectratio}
% Set default figure placement to htbp
\makeatletter
\def\fps@figure{htbp}
\makeatother
\setlength{\emergencystretch}{3em} % prevent overfull lines
\providecommand{\tightlist}{%
  \setlength{\itemsep}{0pt}\setlength{\parskip}{0pt}}
\setcounter{secnumdepth}{5}
\usepackage{booktabs}
\ifluatex
  \usepackage{selnolig}  % disable illegal ligatures
\fi
\usepackage[]{natbib}
\bibliographystyle{apalike}

\title{Advanced R}
\author{Kálmán Abari}
\date{2021-09-27}

\begin{document}
\maketitle

{
\setcounter{tocdepth}{1}
\tableofcontents
}
\hypertarget{introduction}{%
\chapter{Introduction}\label{introduction}}

Welcome to the second book in R Fundamentals series! This second book takes you through how to do manipulation of tabular data and how to create modern graphics in R. We'll primarily be using capabilities from the set of packages called the tidyverse within the book. The book is aimed at beginners to R who understand the basics (check out the \href{https://abarik.github.io/basicr_2020_21_2/}{Basic R}).

\hypertarget{warm-up-exercise}{%
\chapter{Warm-up exercise}\label{warm-up-exercise}}

\begin{quote}
Loftus, S. C. (2021). Basic Statistics with R: Reaching Decisions with Data. Retrieved from \url{https://books.google.hu/books?id=vTASEAAAQBAJ}
\end{quote}

\hypertarget{data-structures}{%
\section{Data structures}\label{data-structures}}

\hypertarget{problems}{%
\subsection{Problems}\label{problems}}

Consider the following set of attributes about the American Film Institute's top-five movies ever from their 2007 list.

\begin{enumerate}
\def\labelenumi{\arabic{enumi}.}
\tightlist
\item
  What code would you use to create a vector named \texttt{Movie} with the values \texttt{Citizen\ Kane}, \texttt{The\ Godfather}, \texttt{Casablanca}, \texttt{Raging\ Bull}, and \texttt{Singing\ in\ the\ Rain}? (Hints: \texttt{object\ \textless{}-\ c()}, Working with character in R)
\item
  What code would you use to create a vector --- giving the year that the movies in Problem 1 were made --- named \texttt{Year} with the values 1941, 1972, 1942, 1980, and 1952?
\item
  What code would you use to create a vector --- giving the run times in minutes of the movies in Problem 1 --- named \texttt{RunTime} with the values 119, 177, 102, 129, and 103?
\item
  What code would you use to find the run times of the movies in hours and save them in a vector called \texttt{RunTimeHours}? (Hints: Numeric tranformation)
\item
  What code would you use to create a data frame named \texttt{MovieInfo} containing the vectors created in Problem 1, Problem 2, and Problem 3? (Hints: \texttt{data.frame()})
\end{enumerate}

\hypertarget{manipulation}{%
\section{Manipulation}\label{manipulation}}

\hypertarget{problems-1}{%
\subsection{Problems}\label{problems-1}}

Suppose we have the following data frame named \texttt{colleges} (\href{data/colleges.xlsx}{download here}):

\begin{tabular}{l|r|r|r}
\hline
College & Employees & TopSalary & MedianSalary\\
\hline
William and Mary & 2104 & 425000 & 56496\\
\hline
Christopher Newport & 922 & 381486 & 47895\\
\hline
George Mason & 4043 & 536714 & 63029\\
\hline
James Madison & 2833 & 428400 & 53080\\
\hline
Longwood & 746 & 328268 & 52000\\
\hline
Norfolk State & 919 & 295000 & 49605\\
\hline
Old Dominion & 2369 & 448272 & 54416\\
\hline
Radford & 1273 & 312080 & 51000\\
\hline
Mary Washington & 721 & 449865 & 53045\\
\hline
Virginia & 7431 & 561099 & 60048\\
\hline
Virginia Commonwealth & 5825 & 503154 & 55000\\
\hline
Virginia Military Institute & 550 & 364269 & 44999\\
\hline
Virginia Tech & 7303 & 500000 & 51656\\
\hline
Virginia State & 761 & 356524 & 55925\\
\hline
\end{tabular}

\begin{enumerate}
\def\labelenumi{\arabic{enumi}.}
\tightlist
\item
  What code would you use to select the first, third, tenth, and twelfth entries in the \texttt{TopSalary} vector from the \texttt{Colleges} data frame? (Hints: Indexing with \texttt{{[}{]}} operator)
\item
  What code would you use to select the elements of the \texttt{MedianSalary} vector where the \texttt{TopSalary} is greater than \$400,000? (Hints: \texttt{d\$MedianSalary{[}d\$TopSalary\textgreater{}400000{]}})
\item
  What code would you use to select the rows of the data frame for colleges with less than or equal to 1000 employees? (Hints: \texttt{d{[}condition,\ {]}})
\item
  What code would you use to select a sample of 5 colleges from this data frame (there are 14 rows)? (Hints: \texttt{d{[}sample(x\ =\ 1:14,\ size\ =\ 5,\ replace\ =\ F),{]}})
\end{enumerate}

Suppose we have the following data frame named Countries (\href{data/countries.xlsx}{download here}):

\begin{tabular}{l|l|r|r|r}
\hline
Nation & Region & Population & PctIncrease & GDPcapita\\
\hline
China & Asia & 1409517397 & 0.4 & 8582\\
\hline
India & Asia & 1339180127 & 1.1 & 1852\\
\hline
United States & North America & 324459463 & 0.7 & 57467\\
\hline
Indonesia & Asia & 263991379 & 1.1 & 3895\\
\hline
Brazil & South America & 209288278 & 0.8 & 10309\\
\hline
Pakistan & Asia & 197015955 & 2.0 & 1629\\
\hline
Nigeria & Africa & 190886311 & 2.6 & 2640\\
\hline
Bangladesh & Asia & 164669751 & 1.1 & 1524\\
\hline
Russia & Europe & 143989754 & 0.0 & 10248\\
\hline
Mexico & North America & 129163276 & 1.3 & 8562\\
\hline
\end{tabular}

\begin{enumerate}
\def\labelenumi{\arabic{enumi}.}
\setcounter{enumi}{4}
\tightlist
\item
  What could would you use to select the rows of the data frame that have GDP per capita less than 10000 and are not in the Asia region?
\item
  What code would you use to select a sample of three nations from this data frame (There are 10 rows)?
\item
  What code would you use to select which nations saw a population percent increase greater that 1.5\%?
\end{enumerate}

Suppose we have the following data frame named Olympics (\href{data/olympics.xlsx}{download here}):

\begin{tabular}{r|l|l|r|r|r|l}
\hline
Year & Type & Host & Competitors & Events & Nations & Leader\\
\hline
1992 & Summer & Spain & 9356 & 257 & 169 & Unified Team\\
\hline
1992 & Winter & France & 1801 & 57 & 64 & Germany\\
\hline
1994 & Winter & Norway & 1737 & 61 & 67 & Russia\\
\hline
1996 & Summer & United States & 10318 & 271 & 197 & United States\\
\hline
1998 & Winter & Japan & 2176 & 68 & 72 & Germany\\
\hline
2000 & Summer & Australia & 10651 & 300 & 199 & United States\\
\hline
2002 & Winter & United States & 2399 & 78 & 78 & Norway\\
\hline
2004 & Summer & Greece & 10625 & 301 & 201 & United States\\
\hline
2006 & Winter & Italy & 2508 & 84 & 80 & Germany\\
\hline
2008 & Summer & China & 10942 & 302 & 204 & China\\
\hline
2010 & Winter & Canada & 2566 & 86 & 82 & Canada\\
\hline
2012 & Summer & United Kingdom & 10768 & 302 & 204 & United States\\
\hline
2014 & Winter & Russia & 2873 & 98 & 88 & Russia\\
\hline
2016 & Summer & Brazil & 11238 & 306 & 207 & United States\\
\hline
2018 & Winter & South Korea & 2922 & 102 & 92 & Norway\\
\hline
\end{tabular}

\begin{enumerate}
\def\labelenumi{\arabic{enumi}.}
\setcounter{enumi}{7}
\tightlist
\item
  What code would you use to select the rows of the data frame where the host nation was also the medal leader?
\item
  What code would you use to select the rows of the data frame where the number of competitors per event is greater than 35?
\item
  What code would you use to select the rows of the data frame where the number of competing nations in the Winter Olympics is at least 80?
\end{enumerate}

\hypertarget{packages}{%
\section{Packages}\label{packages}}

\hypertarget{problems-2}{%
\subsection{Problems}\label{problems-2}}

\begin{enumerate}
\def\labelenumi{\arabic{enumi}.}
\tightlist
\item
  Install the \textbf{Ecdat} package. (Hints: \texttt{install.packages()})
\item
  Say that we previously installed the \textbf{Ecdat} library into R and wanted to call the library to access datasets from it. What code would we use to call the library? (Hints: \texttt{library()})
\item
  Say that we then wanted to call the dataset \texttt{Diamond} from the \textbf{Ecdat} library. What code would we use to load this dataset into R? (Hints: \texttt{data()})
\end{enumerate}

\hypertarget{frequency-and-numerical-exploratory-analyses}{%
\section{Frequency and numerical exploratory analyses}\label{frequency-and-numerical-exploratory-analyses}}

\hypertarget{problems-3}{%
\subsection{Problems}\label{problems-3}}

Load the \texttt{leuk} dataset from the \emph{MASS} library. This dataset is the survival times (\texttt{time}), white blood cell count (\texttt{wbc}), and the presence of a morphologic characteristic of white blood cells (\texttt{ag}).

\begin{enumerate}
\def\labelenumi{\arabic{enumi}.}
\tightlist
\item
  Generate the frequency table for the presence of the morphologic characteristic.
\item
  Find the median and mean for survival time.
\item
  Find the range, IQR, variance, and standard deviation for white blood cell count.
\item
  Find the correlation between white blood cell count and survival time.
\end{enumerate}

Load the \texttt{survey} dataset from the \emph{MASS} library. This dataset contains the survey responses of a class of college students.

\begin{enumerate}
\def\labelenumi{\arabic{enumi}.}
\setcounter{enumi}{4}
\tightlist
\item
  Create the contingency table of whether or not the student smoked (\texttt{Smoke}) and the student's exercise regimen (\texttt{Exer}). (Hints: \texttt{table()}, \texttt{DescTools::Desc()})
\item
  Find the mean and median of the student's heart rate (\texttt{Pulse}). (Hints: \texttt{summary()}, \texttt{DescTools::Desc()}, \texttt{psych::describe()})
\item
  Find the range, IQR, variance, and standard deviation for student age (\texttt{Age}).
\item
  Find the correlation between the span of the student's writing hand (\texttt{Wr.Hnd}) and nonwriting hand (\texttt{NW.Hnd}). (Hints: \texttt{cor()}, \texttt{DescTools::Desc()})
\end{enumerate}

Load the \texttt{Housing} dataset from the \emph{Ecdat} library. This dataset looks at the variables that affect the sales price of houses.

\begin{enumerate}
\def\labelenumi{\arabic{enumi}.}
\setcounter{enumi}{8}
\tightlist
\item
  Create the contingency table of whether or not the house has a recreation room (\texttt{recroom}) and whether or not the house had a full basement (\texttt{fullbase}).
\item
  Find the mean and median of the house's lot size (\texttt{lotsize}).
\item
  Find the range, IQR, variance, and standard deviation for the sales price (\texttt{price}).
\item
  Find the correlation between the sales price of the house (\texttt{price}) and the number of bedrooms (\texttt{bedrooms}).
\end{enumerate}

\hypertarget{graphical-exploratory-analyses}{%
\section{Graphical exploratory analyses}\label{graphical-exploratory-analyses}}

Load the \texttt{Star} dataset from the \emph{Ecdat} library. This dataset looks at the affect on class sizes on student learning.

\begin{enumerate}
\def\labelenumi{\arabic{enumi}.}
\tightlist
\item
  Generate the scatterplot of the student's math score \texttt{tmathssk} and reading score \texttt{treadssk}. (Hints: \texttt{plot()}, \texttt{ggplot()\ +\ geom\_point()})
\item
  Generate the histogram of the years of teaching experience \texttt{totexpk}. (Hints: \texttt{hist()}, \texttt{ggplot()\ +\ geom\_histogram()})
\item
  Create a new variable in the \texttt{Star} dataset called \texttt{totalscore} that is the sum of the student's math score \texttt{tmathssk} and reading score \texttt{treadssk}. (Hints: tranformation)
\item
  Generate a boxplot of the student's total score \texttt{totalscore} split out by the class size type \texttt{classk}. (Hints: \texttt{boxplot()}, \texttt{ggplot()\ +\ geom\_boxplot()})
\end{enumerate}

Load the \texttt{survey} dataset from the \emph{MASS} library. This dataset contains the survey responses of a class of college students.

\begin{enumerate}
\def\labelenumi{\arabic{enumi}.}
\setcounter{enumi}{4}
\tightlist
\item
  Generate the scatterplot of the student's height \texttt{Height} and writing hand span \texttt{Wr.Hnd}.
\item
  Generate the histogram of student age \texttt{Age}.
\item
  Generate a boxplot of the student's heart rate \texttt{Pulse} split out by the student's exercise regimen \texttt{Exer}.
\end{enumerate}

\hypertarget{rmarkdown}{%
\chapter{RMarkdown}\label{rmarkdown}}

RMarkdown is a framework from RStudio for easily combining your code, data, text and interactive charts into both reports and slide decks. RMarkdown is based on Markdown.

\hypertarget{markdown}{%
\section{Markdown}\label{markdown}}

Markdown is a markup language. It is an extremely simple markup language, so it is very popular on the Web and in other application. Markdown is used to format text on GitHub, Reddit, Stack Exchange, and Trello, and in RMarkdown.

Markdown was created by John Gruber and Aaron Swartz in 2004. Markup was designed that a human reader could easily parse the content.

\begin{itemize}
\tightlist
\item
  \href{https://www.markdownguide.org/cheat-sheet/}{Markdown Cheat Sheet}
  A quick reference to the Markdown syntax.
\item
  \href{https://www.markdownguide.org/basic-syntax/}{Basic Syntax}
  The Markdown elements outlined in John Gruber's design document.
\item
  \href{https://www.markdownguide.org/extended-syntax/}{Extended Syntax}
  Advanced features that build on the basic Markdown syntax.
\end{itemize}

\hypertarget{rmarkdown-1}{%
\section{RMarkdown}\label{rmarkdown-1}}

R Markdown understands Pandoc's Markdown, a version of Markdown with more features. \href{https://pandoc.org/MANUAL.html\#pandocs-markdown}{This Pandoc guide} provides and extensive resource for formatting options.

\href{https://rmarkdown.rstudio.com/articles_docx.html}{Happy collaboration with Rmd to docx}

\href{https://ardata-fr.github.io/flextable-book/index.html}{Using the flextable R package}

\hypertarget{additional-resources}{%
\section{Additional Resources}\label{additional-resources}}

\begin{itemize}
\tightlist
\item
  \href{https://bookdown.org/yihui/rmarkdown-cookbook/}{R Markdown Cookbook}
  A comprehensive free online book that contains almost everything you need to know about RMarkdown.
\item
  \href{https://rmd4sci.njtierney.com/}{RMarkdown for Scientists}
\item
  \href{https://support.rstudio.com/hc/en-us/sections/200149716-R-Markdown}{RStudio Articles for RMarkdown}
  RStudio has published a few in-depth how to articles about using RMarkdown.
\item
  \href{https://r4ds.had.co.nz/index.html}{R for Data Science}
  Hadley Wickham provides a great overview of authoring with RMarkdown.
\item
  \href{https://bookdown.org/yihui/rmarkdown/}{R Markdown: The Definitive Guide}
  It contains a large number of technical details, it may serve you better as a reference book than a textbook.
\item
  \href{https://rmarkdown.rstudio.com/lesson-1.html}{Online lesson from RStudio}
\end{itemize}

Markdown is file format.

\url{https://github.com/citation-style-language/styles}

\hypertarget{advanced-data-manipulation}{%
\chapter{Advanced data manipulation}\label{advanced-data-manipulation}}

This chapter focuses exclusively on advanced data manipulation. I therefore assume a basic level of comfort with data manipulation.

\hypertarget{modern-graphics}{%
\chapter{Modern graphics}\label{modern-graphics}}

\hypertarget{tidyverse-r}{%
\chapter{Tidyverse R}\label{tidyverse-r}}

\hypertarget{bioconductor}{%
\chapter{Bioconductor}\label{bioconductor}}

\hypertarget{rna-seq-an-example}{%
\chapter{RNA-Seq (an example)}\label{rna-seq-an-example}}

  \bibliography{book.bib,packages.bib}

\end{document}
